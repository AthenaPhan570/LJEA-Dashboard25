% Options for packages loaded elsewhere
\PassOptionsToPackage{unicode}{hyperref}
\PassOptionsToPackage{hyphens}{url}
%
\documentclass[
]{article}
\usepackage{amsmath,amssymb}
\usepackage{iftex}
\ifPDFTeX
  \usepackage[T1]{fontenc}
  \usepackage[utf8]{inputenc}
  \usepackage{textcomp} % provide euro and other symbols
\else % if luatex or xetex
  \usepackage{unicode-math} % this also loads fontspec
  \defaultfontfeatures{Scale=MatchLowercase}
  \defaultfontfeatures[\rmfamily]{Ligatures=TeX,Scale=1}
\fi
\usepackage{lmodern}
\ifPDFTeX\else
  % xetex/luatex font selection
\fi
% Use upquote if available, for straight quotes in verbatim environments
\IfFileExists{upquote.sty}{\usepackage{upquote}}{}
\IfFileExists{microtype.sty}{% use microtype if available
  \usepackage[]{microtype}
  \UseMicrotypeSet[protrusion]{basicmath} % disable protrusion for tt fonts
}{}
\makeatletter
\@ifundefined{KOMAClassName}{% if non-KOMA class
  \IfFileExists{parskip.sty}{%
    \usepackage{parskip}
  }{% else
    \setlength{\parindent}{0pt}
    \setlength{\parskip}{6pt plus 2pt minus 1pt}}
}{% if KOMA class
  \KOMAoptions{parskip=half}}
\makeatother
\usepackage{xcolor}
\usepackage[margin=1in]{geometry}
\usepackage{color}
\usepackage{fancyvrb}
\newcommand{\VerbBar}{|}
\newcommand{\VERB}{\Verb[commandchars=\\\{\}]}
\DefineVerbatimEnvironment{Highlighting}{Verbatim}{commandchars=\\\{\}}
% Add ',fontsize=\small' for more characters per line
\usepackage{framed}
\definecolor{shadecolor}{RGB}{248,248,248}
\newenvironment{Shaded}{\begin{snugshade}}{\end{snugshade}}
\newcommand{\AlertTok}[1]{\textcolor[rgb]{0.94,0.16,0.16}{#1}}
\newcommand{\AnnotationTok}[1]{\textcolor[rgb]{0.56,0.35,0.01}{\textbf{\textit{#1}}}}
\newcommand{\AttributeTok}[1]{\textcolor[rgb]{0.13,0.29,0.53}{#1}}
\newcommand{\BaseNTok}[1]{\textcolor[rgb]{0.00,0.00,0.81}{#1}}
\newcommand{\BuiltInTok}[1]{#1}
\newcommand{\CharTok}[1]{\textcolor[rgb]{0.31,0.60,0.02}{#1}}
\newcommand{\CommentTok}[1]{\textcolor[rgb]{0.56,0.35,0.01}{\textit{#1}}}
\newcommand{\CommentVarTok}[1]{\textcolor[rgb]{0.56,0.35,0.01}{\textbf{\textit{#1}}}}
\newcommand{\ConstantTok}[1]{\textcolor[rgb]{0.56,0.35,0.01}{#1}}
\newcommand{\ControlFlowTok}[1]{\textcolor[rgb]{0.13,0.29,0.53}{\textbf{#1}}}
\newcommand{\DataTypeTok}[1]{\textcolor[rgb]{0.13,0.29,0.53}{#1}}
\newcommand{\DecValTok}[1]{\textcolor[rgb]{0.00,0.00,0.81}{#1}}
\newcommand{\DocumentationTok}[1]{\textcolor[rgb]{0.56,0.35,0.01}{\textbf{\textit{#1}}}}
\newcommand{\ErrorTok}[1]{\textcolor[rgb]{0.64,0.00,0.00}{\textbf{#1}}}
\newcommand{\ExtensionTok}[1]{#1}
\newcommand{\FloatTok}[1]{\textcolor[rgb]{0.00,0.00,0.81}{#1}}
\newcommand{\FunctionTok}[1]{\textcolor[rgb]{0.13,0.29,0.53}{\textbf{#1}}}
\newcommand{\ImportTok}[1]{#1}
\newcommand{\InformationTok}[1]{\textcolor[rgb]{0.56,0.35,0.01}{\textbf{\textit{#1}}}}
\newcommand{\KeywordTok}[1]{\textcolor[rgb]{0.13,0.29,0.53}{\textbf{#1}}}
\newcommand{\NormalTok}[1]{#1}
\newcommand{\OperatorTok}[1]{\textcolor[rgb]{0.81,0.36,0.00}{\textbf{#1}}}
\newcommand{\OtherTok}[1]{\textcolor[rgb]{0.56,0.35,0.01}{#1}}
\newcommand{\PreprocessorTok}[1]{\textcolor[rgb]{0.56,0.35,0.01}{\textit{#1}}}
\newcommand{\RegionMarkerTok}[1]{#1}
\newcommand{\SpecialCharTok}[1]{\textcolor[rgb]{0.81,0.36,0.00}{\textbf{#1}}}
\newcommand{\SpecialStringTok}[1]{\textcolor[rgb]{0.31,0.60,0.02}{#1}}
\newcommand{\StringTok}[1]{\textcolor[rgb]{0.31,0.60,0.02}{#1}}
\newcommand{\VariableTok}[1]{\textcolor[rgb]{0.00,0.00,0.00}{#1}}
\newcommand{\VerbatimStringTok}[1]{\textcolor[rgb]{0.31,0.60,0.02}{#1}}
\newcommand{\WarningTok}[1]{\textcolor[rgb]{0.56,0.35,0.01}{\textbf{\textit{#1}}}}
\usepackage{graphicx}
\makeatletter
\newsavebox\pandoc@box
\newcommand*\pandocbounded[1]{% scales image to fit in text height/width
  \sbox\pandoc@box{#1}%
  \Gscale@div\@tempa{\textheight}{\dimexpr\ht\pandoc@box+\dp\pandoc@box\relax}%
  \Gscale@div\@tempb{\linewidth}{\wd\pandoc@box}%
  \ifdim\@tempb\p@<\@tempa\p@\let\@tempa\@tempb\fi% select the smaller of both
  \ifdim\@tempa\p@<\p@\scalebox{\@tempa}{\usebox\pandoc@box}%
  \else\usebox{\pandoc@box}%
  \fi%
}
% Set default figure placement to htbp
\def\fps@figure{htbp}
\makeatother
\setlength{\emergencystretch}{3em} % prevent overfull lines
\providecommand{\tightlist}{%
  \setlength{\itemsep}{0pt}\setlength{\parskip}{0pt}}
\setcounter{secnumdepth}{-\maxdimen} % remove section numbering
\usepackage{bookmark}
\IfFileExists{xurl.sty}{\usepackage{xurl}}{} % add URL line breaks if available
\urlstyle{same}
\hypersetup{
  pdftitle={server.R},
  pdfauthor={tuong},
  hidelinks,
  pdfcreator={LaTeX via pandoc}}

\title{server.R}
\author{tuong}
\date{2025-06-17}

\begin{document}
\maketitle

\begin{Shaded}
\begin{Highlighting}[]
\CommentTok{\#}
\CommentTok{\# This is the server logic of a Shiny web application. You can run the}
\CommentTok{\# application by clicking \textquotesingle{}Run App\textquotesingle{} above.}
\CommentTok{\#}
\CommentTok{\# Find out more about building applications with Shiny here:}
\CommentTok{\#}
\CommentTok{\#    https://shiny.posit.co/}
\CommentTok{\#}
\FunctionTok{library}\NormalTok{(shiny)}
\FunctionTok{library}\NormalTok{(dplyr)}
\end{Highlighting}
\end{Shaded}

\begin{verbatim}
## 
## Attaching package: 'dplyr'
\end{verbatim}

\begin{verbatim}
## The following objects are masked from 'package:stats':
## 
##     filter, lag
\end{verbatim}

\begin{verbatim}
## The following objects are masked from 'package:base':
## 
##     intersect, setdiff, setequal, union
\end{verbatim}

\begin{Shaded}
\begin{Highlighting}[]
\FunctionTok{library}\NormalTok{(tidyverse)}
\end{Highlighting}
\end{Shaded}

\begin{verbatim}
## -- Attaching core tidyverse packages ------------------------ tidyverse 2.0.0 --
## v forcats   1.0.0     v readr     2.1.5
## v ggplot2   3.5.1     v stringr   1.5.1
## v lubridate 1.9.4     v tibble    3.2.1
## v purrr     1.0.4     v tidyr     1.3.1
\end{verbatim}

\begin{verbatim}
## -- Conflicts ------------------------------------------ tidyverse_conflicts() --
## x dplyr::filter() masks stats::filter()
## x dplyr::lag()    masks stats::lag()
## i Use the conflicted package (<http://conflicted.r-lib.org/>) to force all conflicts to become errors
\end{verbatim}

\begin{Shaded}
\begin{Highlighting}[]
\FunctionTok{library}\NormalTok{(ggplot2)}

\NormalTok{server }\OtherTok{\textless{}{-}} \ControlFlowTok{function}\NormalTok{(input, output, session) \{}
  \CommentTok{\# 1. Reading in the non edited/cleaned data sets}
\NormalTok{  getBioData }\OtherTok{\textless{}{-}} \FunctionTok{reactive}\NormalTok{(\{}
\NormalTok{    bio\_data }\OtherTok{\textless{}{-}} \FunctionTok{read.csv}\NormalTok{(}\StringTok{"C:/Users/tuong/OneDrive/Desktop/bio\_data.csv"}\NormalTok{)}
    
    \CommentTok{\# Rename columns with phys\_ prefix}
\NormalTok{    bio\_data }\OtherTok{\textless{}{-}}\NormalTok{ bio\_data }\SpecialCharTok{\%\textgreater{}\%}
      \FunctionTok{rename\_with}\NormalTok{(}\AttributeTok{.cols =} \FunctionTok{c}\NormalTok{(}\StringTok{\textasciigrave{}}\AttributeTok{ncbi}\StringTok{\textasciigrave{}}\NormalTok{, }\StringTok{\textasciigrave{}}\AttributeTok{bioclassification}\StringTok{\textasciigrave{}}\NormalTok{),}
                  \AttributeTok{.fn =} \SpecialCharTok{\textasciitilde{}} \FunctionTok{paste0}\NormalTok{(}\StringTok{"bio\_"}\NormalTok{, .x))}
    
    \FunctionTok{return}\NormalTok{(bio\_data)}
\NormalTok{  \})}
  
\NormalTok{  getHabitatData }\OtherTok{\textless{}{-}} \FunctionTok{reactive}\NormalTok{(\{}
    \FunctionTok{read.csv}\NormalTok{(}\StringTok{"C:/Users/tuong/OneDrive/Desktop/hab\_data.csv"}\NormalTok{)}
\NormalTok{  \})}
  
\NormalTok{  getChemicalData }\OtherTok{\textless{}{-}} \FunctionTok{reactive}\NormalTok{(\{}
    \FunctionTok{read.csv}\NormalTok{(}\StringTok{"C:/Users/tuong/OneDrive/Desktop/25vwin.csv"}\NormalTok{)}
\NormalTok{  \})}
  
  \CommentTok{\# Observe: reactive expression in that it can read reactive values and call reactive expressions, and will automatically re{-}execute when those dependencies change. }
  
  \CommentTok{\# 2. Update based on user input on each tab}
  \FunctionTok{observe}\NormalTok{(\{}
    \FunctionTok{req}\NormalTok{(input}\SpecialCharTok{$}\NormalTok{tabs)}
    
    \ControlFlowTok{if}\NormalTok{ (input}\SpecialCharTok{$}\NormalTok{tabs }\SpecialCharTok{==} \StringTok{"bio"}\NormalTok{) \{}
      \FunctionTok{updateSelectInput}\NormalTok{(session, }\StringTok{"site\_name"}\NormalTok{,  }\CommentTok{\# matches selectInput("site\_name", ...)}
                        \AttributeTok{choices =} \FunctionTok{unique}\NormalTok{(}\FunctionTok{getBioData}\NormalTok{()}\SpecialCharTok{$}\NormalTok{site.name))}
      
\NormalTok{    \} }\ControlFlowTok{else} \ControlFlowTok{if}\NormalTok{ (input}\SpecialCharTok{$}\NormalTok{tabs }\SpecialCharTok{==} \StringTok{"habitat"}\NormalTok{) \{}
      \FunctionTok{updateSelectInput}\NormalTok{(session, }\StringTok{"site\_name\_hab"}\NormalTok{,  }\CommentTok{\# matches selectInput("site\_name\_hab", ...)}
                        \AttributeTok{choices =} \FunctionTok{unique}\NormalTok{(}\FunctionTok{getHabitatData}\NormalTok{()}\SpecialCharTok{$}\NormalTok{name))}
      
\NormalTok{    \} }\ControlFlowTok{else} \ControlFlowTok{if}\NormalTok{ (input}\SpecialCharTok{$}\NormalTok{tabs }\SpecialCharTok{==} \StringTok{"chemical"}\NormalTok{) \{}
      \FunctionTok{updateSelectInput}\NormalTok{(session, }\StringTok{"site\_name\_chem"}\NormalTok{,  }\CommentTok{\# matches selectInput("site\_name\_chem", ...)}
                        \AttributeTok{choices =} \FunctionTok{unique}\NormalTok{(}\FunctionTok{getChemicalData}\NormalTok{()}\SpecialCharTok{$}\NormalTok{site.name)) }\CommentTok{\# matches column name of dataset}
\NormalTok{    \}}
\NormalTok{  \})}
  
  \CommentTok{\# 3. Attempt at trying to edit the data based on user input}
  
  \CommentTok{\# Filtered Biological Data}
\NormalTok{  filteredBioData }\OtherTok{\textless{}{-}} \FunctionTok{reactive}\NormalTok{(\{}
    \FunctionTok{req}\NormalTok{(input}\SpecialCharTok{$}\NormalTok{site\_name)}
    \FunctionTok{getBioData}\NormalTok{() }\SpecialCharTok{\%\textgreater{}\%}\NormalTok{ dplyr}\SpecialCharTok{::}\FunctionTok{filter}\NormalTok{(site.name }\SpecialCharTok{==}\NormalTok{ input}\SpecialCharTok{$}\NormalTok{site\_name)}
\NormalTok{  \})}
  
  
  \CommentTok{\# Filtered Habitat Data}
\NormalTok{  filteredHabitatData }\OtherTok{\textless{}{-}} \FunctionTok{reactive}\NormalTok{(\{}
    \FunctionTok{req}\NormalTok{(input}\SpecialCharTok{$}\NormalTok{site\_name\_hab)}
    \FunctionTok{getHabitatData}\NormalTok{() }\SpecialCharTok{\%\textgreater{}\%}\NormalTok{ dplyr}\SpecialCharTok{::}\FunctionTok{filter}\NormalTok{(name }\SpecialCharTok{==}\NormalTok{ input}\SpecialCharTok{$}\NormalTok{site\_name\_hab)}
\NormalTok{  \})}
  
  \CommentTok{\# Filtered Chemical Data}
\NormalTok{  filteredChemicalData }\OtherTok{\textless{}{-}} \FunctionTok{reactive}\NormalTok{(\{}
    \FunctionTok{req}\NormalTok{(input}\SpecialCharTok{$}\NormalTok{site\_name\_chem)}
    \FunctionTok{getChemicalData}\NormalTok{() }\SpecialCharTok{\%\textgreater{}\%}\NormalTok{ dplyr}\SpecialCharTok{::}\FunctionTok{filter}\NormalTok{(site.name }\SpecialCharTok{==}\NormalTok{ input}\SpecialCharTok{$}\NormalTok{site\_name\_chem)}
\NormalTok{  \})}
  
  
  \CommentTok{\# Output Barplots (change all to ggplot)}
  
\NormalTok{  output}\SpecialCharTok{$}\NormalTok{bioPlot }\OtherTok{\textless{}{-}} \FunctionTok{renderPlot}\NormalTok{(\{}
\NormalTok{    data }\OtherTok{\textless{}{-}} \FunctionTok{filteredBioData}\NormalTok{()}
    
\NormalTok{    bio\_long }\OtherTok{\textless{}{-}}\NormalTok{ data }\SpecialCharTok{\%\textgreater{}\%}
\NormalTok{      tidyr}\SpecialCharTok{::}\FunctionTok{pivot\_longer}\NormalTok{(}\AttributeTok{cols =} \FunctionTok{starts\_with}\NormalTok{(}\StringTok{"bio\_"}\NormalTok{), }\AttributeTok{names\_to =} \StringTok{"Variable"}\NormalTok{, }\AttributeTok{values\_to =} \StringTok{"Value"}\NormalTok{)}
    
    \CommentTok{\# fill = variable (gives color)}
    \FunctionTok{ggplot}\NormalTok{(bio\_long, }\FunctionTok{aes}\NormalTok{(}\AttributeTok{x =}\NormalTok{ site.name, }\AttributeTok{y =}\NormalTok{ Value, }\AttributeTok{fill =}\NormalTok{ Variable)) }\SpecialCharTok{+}
      \FunctionTok{geom\_col}\NormalTok{(}\AttributeTok{position =} \StringTok{"dodge"}\NormalTok{) }\SpecialCharTok{+}
      \FunctionTok{labs}\NormalTok{(}\AttributeTok{title =} \StringTok{"Biological Data by Lake James Site"}\NormalTok{, }\AttributeTok{x =} \StringTok{"Site Name"}\NormalTok{, }\AttributeTok{y =} \StringTok{"Value"}\NormalTok{) }\SpecialCharTok{+}
      \FunctionTok{theme\_minimal}\NormalTok{()}
\NormalTok{  \})}
  
  \CommentTok{\# Biological Tab Outputs}
  
\NormalTok{  output}\SpecialCharTok{$}\NormalTok{bioSummary }\OtherTok{\textless{}{-}}\NormalTok{ DT}\SpecialCharTok{::}\FunctionTok{renderDataTable}\NormalTok{(\{}
\NormalTok{    data }\OtherTok{\textless{}{-}} \FunctionTok{filteredBioData}\NormalTok{()}
\NormalTok{    DT}\SpecialCharTok{::}\FunctionTok{datatable}\NormalTok{(}\FunctionTok{summary}\NormalTok{(data), }\AttributeTok{caption =} \StringTok{"Biological Data Summary"}\NormalTok{)}
\NormalTok{  \})}
  
  \CommentTok{\# Habitat Tab Outputs}
  
  
\NormalTok{  output}\SpecialCharTok{$}\NormalTok{habitatPlot }\OtherTok{\textless{}{-}} \FunctionTok{renderPlot}\NormalTok{(\{}
\NormalTok{    data }\OtherTok{\textless{}{-}} \FunctionTok{filteredHabitatData}\NormalTok{()}
    \FunctionTok{hist}\NormalTok{(data[[}\DecValTok{1}\NormalTok{]], }\AttributeTok{main =} \StringTok{"Habitat Data Histogram"}\NormalTok{, }\AttributeTok{xlab =} \FunctionTok{names}\NormalTok{(data)[}\DecValTok{1}\NormalTok{])}
\NormalTok{  \})}
  
\NormalTok{  output}\SpecialCharTok{$}\NormalTok{habitatSummary }\OtherTok{\textless{}{-}}\NormalTok{ DT}\SpecialCharTok{::}\FunctionTok{renderDataTable}\NormalTok{(\{}
\NormalTok{    data }\OtherTok{\textless{}{-}} \FunctionTok{filteredHabitatData}\NormalTok{()}
\NormalTok{    DT}\SpecialCharTok{::}\FunctionTok{datatable}\NormalTok{(}\FunctionTok{summary}\NormalTok{(data), }\AttributeTok{caption =} \StringTok{"Habitat Data Summary"}\NormalTok{)}
\NormalTok{  \})}
  
  \CommentTok{\# Chemical Tab Outputs}
\NormalTok{  output}\SpecialCharTok{$}\NormalTok{chemPlot }\OtherTok{\textless{}{-}} \FunctionTok{renderPlot}\NormalTok{(\{}
\NormalTok{    data }\OtherTok{\textless{}{-}} \FunctionTok{filteredChemicalData}\NormalTok{()}
    \FunctionTok{hist}\NormalTok{(data[[}\DecValTok{1}\NormalTok{]], }\AttributeTok{main =} \StringTok{"Chemical Data Histogram"}\NormalTok{, }\AttributeTok{xlab =} \FunctionTok{names}\NormalTok{(data)[}\DecValTok{1}\NormalTok{])}
\NormalTok{  \})}
  
\NormalTok{  output}\SpecialCharTok{$}\NormalTok{chemSummary }\OtherTok{\textless{}{-}}\NormalTok{ DT}\SpecialCharTok{::}\FunctionTok{renderDataTable}\NormalTok{(\{}
\NormalTok{    data }\OtherTok{\textless{}{-}} \FunctionTok{filteredChemicalData}\NormalTok{()}
\NormalTok{    DT}\SpecialCharTok{::}\FunctionTok{datatable}\NormalTok{(}\FunctionTok{summary}\NormalTok{(data), }\AttributeTok{caption =} \StringTok{"Chemical Data Summary"}\NormalTok{)}
\NormalTok{  \})}
\NormalTok{\}}
\end{Highlighting}
\end{Shaded}


\end{document}
